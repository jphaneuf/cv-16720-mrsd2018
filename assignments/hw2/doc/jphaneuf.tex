\documentclass[12pt]{article}
\usepackage[english]{babel}
\usepackage{float}
\usepackage[margin=1in]{geometry}
\usepackage{graphicx}
%\usepackage[toc,page]{appendix}
\graphicspath{ {./doc/img/} }
\newcommand{\rpm}{\raisebox{.2ex}{$\scriptstyle\pm$}} 
\usepackage{listings}
\usepackage{xcolor}
\usepackage{indentfirst}
\usepackage{caption}
\usepackage[final]{pdfpages}


\begin{document}

\title{Joe Phaneuf \\ Computer Vision 16-720 Spring 2018 Homework 2 \\ Feb. 22, 2018 }
\date{}
\author{}
\maketitle

\newpage


%\stepcounter{section}
%%%%%%%%%%%%%%%%%%%%%%%%%%%%%%%%%%%%%%%%%%%%%%%%%%%%%%%%%%%%%%%%%%%%%%%%%%%%%%%%
%%%%%%%%%%%%%%%%%%%%%%%%%%%%%%%%%%%%%%%%%%%%%%%%%%%%%%%%%%%%%%%%%%%%%%%%%%%%%%%%
\section{Q1}
\subsection{Q1.1}

The filter bank contains four types of filters at different scales. The first filter type is Gaussian, which suppresses high frequency content. The next filter type is Laplacian of Gaussian , which responds to sharp edges. The next two filter types are derivative filters, which will respond to vertical and horizontal edges respectively.
Figure \ref{fig:test_image} shows a test image for the filters, Figure shows \ref{fig:filters} the filters themselves, while Figure \ref{fig:filter_responses} shows the filtered images.

\begin{figure}[H]
\centering
\includegraphics[page=1,width=0.4\textwidth]{test}
\caption{Test image}    
\label{fig:test_image}
\end{figure}   

\begin{figure}[H]
\centering
\includegraphics[page=1,width=0.75\textwidth]{q1_filters}
\caption{Filters}    
\label{fig:filters}
\end{figure}   

\begin{figure}[H]
\centering
\includegraphics[page=1,width=0.75\textwidth]{q1_responses}
\caption{Filter responses}    
\label{fig:filter_responses}
\end{figure}   

%%%%%%%%%%%%%%%%%%%%%%%%%%%%%%%%%%%%%%%%%%%%%%%%%%%%%%%%%%%%%%%%%%%%%%%%%%%%%%%%
%%%%%%%%%%%%%%%%%%%%%%%%%%%%%%%%%%%%%%%%%%%%%%%%%%%%%%%%%%%%%%%%%%%%%%%%%%%%%%%%
\newpage
\subsection{Q1.2}
The extractFilterResponses function converts an input image to the CIE LAB color space, and subsequently applies the filter bank described above to each channel (L A B).
The CIE LAB color space is chosen for this application as the image intensity is stored in a separate channel (L) from color information (A B). In RGB color space (default for most input images), image intensity is a combination of all channels, and image processing techniques are therefore highly sensitive to the lighting conditions in which the image was captured. The CIE LAB color space color channels will be more similar for images of the same object in different lighting conditions. Figure \ref{fig:extract_filter_responses} shows a sample image along with one feature filter applied to the L, A, and B channels.

\begin{figure}[H]
\centering
\includegraphics[page=1,width=0.75\textwidth]{q12_extract_filters}
\caption{Filter responses on LAB colored images}    
\label{fig:extract_filter_responses}
\end{figure}   

%%%%%%%%%%%%%%%%%%%%%%%%%%%%%%%%%%%%%%%%%%%%%%%%%%%%%%%%%%%%%%%%%%%%%%%%%%%%%%%%
%%%%%%%%%%%%%%%%%%%%%%%%%%%%%%%%%%%%%%%%%%%%%%%%%%%%%%%%%%%%%%%%%%%%%%%%%%%%%%%%
\newpage
\subsection{Q1.3}

Figure \ref{fig:harris_corners} shows my Harris corner detector applied to three images.

\begin{figure}[H]
\centering
\includegraphics[page=1,width=1\textwidth]{q13_harris}
\caption{Harris corners}    
\label{fig:harris_corners}
\end{figure}   

%%%%%%%%%%%%%%%%%%%%%%%%%%%%%%%%%%%%%%%%%%%%%%%%%%%%%%%%%%%%%%%%%%%%%%%%%%%%%%%%
%%%%%%%%%%%%%%%%%%%%%%%%%%%%%%%%%%%%%%%%%%%%%%%%%%%%%%%%%%%%%%%%%%%%%%%%%%%%%%%%
\newpage
\section{Q2}
\subsection{Q2.1}

Figure \ref{fig:wordmapharris} shows the word map produced on 3 images of airport and rain forest class using the Harris corner generated dictionary.
Figure \ref{fig:wordmaprandom} shows the word map produced on 3 images of airport and rain forest class using the random corner generated dictionary.
The goal of viewing these images is to capture semantic labeling on swatches of an image, so "better" means more contiguous space logically grouped together (i.e. all pixels of a swath of grass have the same label.) The word maps from the Harris dictionary appear to accomplish this better than the word maps from the random dictionary.

\begin{figure}[H]
\centering
\includegraphics[page=1,width=1\textwidth]{q21_wordmap_harris}
\caption{Wordmap Harris corners corners}    
\label{fig:wordmapharris}
\end{figure}   


\begin{figure}[H]
\centering
\includegraphics[page=1,width=1\textwidth]{q21_wordmap_random}
\caption{Wordmap random corners}    
\label{fig:wordmaprandom}
\end{figure}   

%%%%%%%%%%%%%%%%%%%%%%%%%%%%%%%%%%%%%%%%%%%%%%%%%%%%%%%%%%%%%%%%%%%%%%%%%%%%%%%%
%%%%%%%%%%%%%%%%%%%%%%%%%%%%%%%%%%%%%%%%%%%%%%%%%%%%%%%%%%%%%%%%%%%%%%%%%%%%%%%%
\newpage
\section{Q3}
\subsection{Q3.2}

The tables below show visual bag of words accuracies and confusion matrices using an alpha of 300 and k of 200 for the permutations of Harris corner / random point and Euclidean / chi2 distance.  
Using chi2 distance outperforms Euclidean distance in both cases, which makes sense as chi2 distance is more of a measure of similarity than a 'spatial' Euclidean distance.
Harris corner point selection outperforms random point selection using Euclidean distance , but not using chi2. I suspect the advantages of Harris vs. random point selection are highly dependent on image type, where images with many definitive corners ( such as airports ) may be better sampled using Harris corners, while images with fewer meaningful corners (such as rain forests) may be better sampled at random.

Tables \ref{table:harrischi2} , \ref{table:harriseuc} , \ref{table:randomchi2} , and \ref{table:randomeuc} show confusion matrices for each permutation described above, where each row represents the actual label and each column represents the predicted label. The total accuracy is listed below each table.






\subsubsection{confusion matrix harris chi2}
\begin{table}[H]
\centering
\begin{tabular}{c c c c c c c c}
    12 &     2 &     4 &     0 &     0 &     0 &     0 &     2 \\
     3 &    12 &     2 &     0 &     2 &     1 &     0 &     0 \\
     2 &     3 &    11 &     1 &     3 &     0 &     0 &     0 \\
     1 &     2 &     1 &     9 &     0 &     1 &     6 &     0 \\
     0 &     3 &     2 &     1 &    10 &     0 &     4 &     0 \\
     1 &     1 &     3 &     0 &     0 &    10 &     3 &     2 \\
     5 &     1 &     1 &     3 &     1 &     0 &     7 &     2 \\
     4 &     0 &     0 &     2 &     0 &     0 &     1 &    13 
\end{tabular}
\caption{confusion matrix harris chi2}
\label{table:harrischi2}
\end{table}


accuracy = 0.5250


\subsubsection{confusion matrix harris Euclidean}
\begin{table}[H]
\centering
\begin{tabular}{c c c c c c c c}
    12 &     2 &     3 &     0 &     0 &     0 &     1 &     2 \\
     5 &    12 &     2 &     0 &     0 &     1 &     0 &     0 \\
     3 &     5 &    10 &     1 &     1 &     0 &     0 &     0 \\
     2 &     1 &     3 &     7 &     1 &     1 &     5 &     0 \\
     0 &     6 &     1 &     3 &     6 &     0 &     4 &     0 \\
     1 &     2 &     1 &     3 &     2 &     6 &     3 &     2 \\
     4 &     1 &     1 &     5 &     0 &     1 &     6 &     2 \\
     0 &     0 &     1 &     2 &     0 &     0 &     0 &    17 
\end{tabular}
\caption{confusion matrix harris Euclidean}
\label{table:harriseuc}
\end{table}


accuracy = 0.4750


\subsubsection{confusion matrix random chi2}
\begin{table}[H]
\centering
\begin{tabular}{c c c c c c c c}
    12 &     1 &     4 &     1 &     0 &     0 &     0 &     2 \\
     5 &    13 &     1 &     0 &     0 &     1 &     0 &     0 \\
     3 &     4 &    11 &     0 &     1 &     0 &     1 &     0 \\
     2 &     1 &     2 &     8 &     0 &     2 &     5 &     0 \\
     0 &     3 &     3 &     0 &    11 &     0 &     3 &     0 \\
     2 &     0 &     3 &     0 &     1 &     8 &     3 &     3 \\
     1 &     2 &     4 &     4 &     0 &     0 &     8 &     1 \\
     2 &     0 &     0 &     3 &     0 &     0 &     0 &    15 \\
\end{tabular}
\caption{confusion matrix random chi2}
\label{table:randomchi2}
\end{table}

accuracy = 0.5375

\subsubsection{confusion matrix random Euclidean}
\begin{table}[H]
\centering
\begin{tabular}{c c c c c c c c}
    11 &     2 &     2 &     1 &     0 &     0 &     1 &     3 \\
     6 &     9 &     2 &     0 &     1 &     2 &     0 &     0 \\
     4 &     7 &     6 &     1 &     2 &     0 &     0 &     0 \\
     4 &     1 &     2 &     6 &     1 &     2 &     4 &     0 \\
     0 &     4 &     2 &     1 &    10 &     0 &     3 &     0 \\
     2 &     2 &     1 &     2 &     2 &     7 &     2 &     2 \\
     1 &     4 &     1 &     3 &     1 &     2 &     6 &     2 \\
     4 &     0 &     0 &     2 &     0 &     0 &     0 &    14 
\end{tabular}
\caption{confusion matrix random Euclidean}
\label{table:randomeuc}
\end{table}

accuracy =  0.4313


\end{document}
