\documentclass[12pt]{article}
\usepackage[english]{babel}
\usepackage{float}
\usepackage[margin=1in]{geometry}
\usepackage{graphicx}
%\usepackage[toc,page]{appendix}
\graphicspath{ {./img/} }
\newcommand{\rpm}{\raisebox{.2ex}{$\scriptstyle\pm$}} 
\usepackage{listings}
\usepackage{xcolor}
\usepackage{indentfirst}
\usepackage{caption}
\usepackage[final]{pdfpages}
\usepackage{amsmath}
\usepackage[normalem]{ulem}



\begin{document}

\title{Joe Phaneuf \\ Computer Vision 16-720 Spring 2018 Homework 6 \\ Apr. 28 2018 }
\date{}
\author{}
\maketitle

\newpage


%\stepcounter{section}
%%%%%%%%%%%%%%%%%%%%%%%%%%%%%%%%%%%%%%%%%%%%%%%%%%%%%%%%%%%%%%%%%%%%%%%%%%%%%%%%
%%%%%%%%%%%%%%%%%%%%%%%%%%%%%%%%%%%%%%%%%%%%%%%%%%%%%%%%%%%%%%%%%%%%%%%%%%%%%%%%
\section{Q1}
\subsection{Q1.1}
For Lucas-Kanade Tracking, we aim to iteratively compute an optimal image transformation. At each iteration, the optimal change in transform parameters $\Delta p$ is found by linearization as seen in equation \ref{eq:iteration}.

\begin{equation}
\label{eq:iteration}
\Delta p ^{*} = argmin_{\Delta p} \sum_{x}
\begin{bmatrix}
\nabla I \frac { \partial W } { \partial p } \Delta p - \{ T ( x ) - I ( W ( x ; p ) ) \}
\end{bmatrix}^{2}
\end{equation}

Where $I$ is a target image, $\nabla I$ is the image gradient, $W$ is the image transformation function, $\frac{ \partial W } { \partial p }$ is the Jacobian representing the change in pixel position with respect to transform parameters $p$, and $T$ is the template image.

Equation \ref{eq:iteration} can be set up in normal equation form $argmin_{\Delta p} || A \Delta p - b ||_{2}^{2}$ where $A$ and $b$ are:
\begin{equation}
\label{eq:A}
A =
\nabla I \frac { \partial W } { \partial p }
\end{equation}
\begin{equation}
\label{eq:A}
b = T ( x ) - I ( W ( x ; p ) )
\end{equation}

This method banks on a few assumptions to obtain a unique $\Delta p$. First, $A^{T} A$ must be invertible for a solution to be obtained. Also, the eigen values of $A^{T} A$, $\lambda_{1}$ and $\lambda_{2}$, must be sufficiently large, and well conditioned ( $\lambda_{1}$ not significantly larger than $\lambda_{2}$ ). In human terms, this corresponds to the existence of corners in the the image, which allows for tracking.


\end{document}
